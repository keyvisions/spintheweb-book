% Appendix: WBLL Token Reference

\chapter{WBLL Token Reference}
\label{app:wbll-tokens}

This appendix provides a complete reference for WBLL tokens.

% Inline WBLL syntax helper for use inside tables and running text
\newcommand{\wballsyn}[1]{\lstinline[language=WBLL,basicstyle=\ttfamily\small]!#1!}

% General token syntax and notation
\noindent\textbf{General syntax:} \wballsyn{<token>('arg1[;arg2][;arg3]...')}

\noindent Brackets denote optional arguments:
\begin{itemize}
  \item No arguments: \wballsyn{e}
  \item One argument: \wballsyn{e('arg1')}
  \item Two arguments: \wballsyn{e('arg1;arg2')}
  \item Three arguments: \wballsyn{e('arg1;arg2;arg3')}
\end{itemize}
If you need to skip a middle argument, use an empty segment: e.g., \wballsyn{e(';username')} omits \textit{format}.

\par\smallskip
\textit{Formatting note:} the \textit{format} argument is a numeric pattern when the value is numeric (e.g., \texttt{€ \#,\#\#0.00}, \texttt{\#,\#\#0}, \texttt{0}). For non numeric values the argument format can be any of the HTML input types, it defaults to text.
\par\smallskip
\textit{Cursor movement policy:} \textbf{editing tokens} (form controls such as \texttt{h}, \texttt{e}, \texttt{w}, \texttt{m}) advance the field cursor when the \emph{name} is not specified; \textbf{non-editing tokens} advance only when they implicitly consume the active field \emph{value}.

To keep token docs uniform and readable, we use a two-column layout: labels on the left and content (with code and output) on the right.

% Consistent spacing and styling for token boxes
\cbset{tokenbox/.style={interior hidden,colframe=black!15,boxrule=0.5pt,arc=2pt,before skip=2\baselineskip,after skip=2\baselineskip}}

% Example recordset used in examples
\begin{tcolorbox}[tokenbox]
{\small\setstretch{1.05}%
\textbf{Example recordset} (fields in order): \texttt{price}, \texttt{username}, \texttt{city}.

\medskip
\begin{tabularx}{\linewidth}{@{}l l l@{}}
\textbf{price} & \textbf{username} & \textbf{city} \\
\hline
1234.5 & alice & Milan \\
99 & bob & Paris \\
0 & carol & Berlin \\
\end{tabularx}

\medskip
\textit{Assumptions for examples:} unless stated otherwise,
(1) examples evaluate the first row;
(2) the field cursor starts at the first field (\texttt{price});
(3) when currency output like “€ 1{,}234.50” appears, it is the formatted \texttt{price} of row 1 using the pattern \texttt{€ \#,\#\#0.00}.
}
\end{tcolorbox}

% /* */ — multiline comment (non-editing)
\begin{tcolorbox}[tokenbox]
{\small\setstretch{1.05}%
\begin{tabularx}{\linewidth}{@{}p{0.18\linewidth} X@{}}
\textbf{Token} & \wballsyn{/* comment text */} \\
\hline
\textbf{Description} & Multiline comment. The interpreter ignores everything between \texttt{/*} and \texttt{*/}. Can span multiple lines. Produces no output and does \textbf{not} move the field cursor. \\
\hline
\textbf{Example} & Code and output below. \\
\end{tabularx}}

\vspace{0.3em}
\lstset{language=WBLL}
\begin{lstlisting}
lf /* comment
   spanning
   lines */
\rl('UserName')f
\end{lstlisting}
\textit{Renders:}
\lstset{language=HTML}
\begin{lstlisting}
<label>price</label>1234.5<br><label>UserName</label>alice
\end{lstlisting}
\end{tcolorbox}

% // — line comment (non-editing)
\begin{tcolorbox}[tokenbox]
{\small\setstretch{1.05}%
\begin{tabularx}{\linewidth}{@{}p{0.18\linewidth} X@{}}
\textbf{Token} & \wballsyn{// comment text} \\
\hline
\textbf{Description} & Line comment. The interpreter ignores everything from \textbf{//} to the end of the line. Produces no output and does \textbf{not} move the field cursor. \\
\hline
\textbf{Example} & Code and output below. \\
\end{tabularx}}

\vspace{0.3em}
\lstset{language=WBLL}
\begin{lstlisting}
t('Hello') // inline comment after code
\r
// full-line comment
t('World')
\end{lstlisting}
\textit{Renders:}
\lstset{language=HTML}
\begin{lstlisting}
Hello<br>World
\end{lstlisting}
\end{tcolorbox}

% < — move field cursor backward
\begin{tcolorbox}[tokenbox]
{\small\setstretch{1.05}%
\begin{tabularx}{\linewidth}{@{}p{0.18\linewidth} X@{}}
\textbf{Token} & \wballsyn{<} \\
\hline
\textbf{Description} & Moves the field cursor back by one position. Produces no output. If the cursor is already at the first field, this operation is a no-op. \\
\hline
\textbf{Example} & Code and output below. \\
\end{tabularx}}

\vspace{0.3em}
\lstset{language=WBLL}
\begin{lstlisting}
>>><f
\end{lstlisting}
\textit{Renders:}
\lstset{language=HTML}
\begin{lstlisting}
Milan
\end{lstlisting}
\end{tcolorbox}

% > — advance field cursor
\begin{tcolorbox}[tokenbox]
{\small\setstretch{1.05}%
\begin{tabularx}{\linewidth}{@{}p{0.18\linewidth} X@{}}
\textbf{Token} & \wballsyn{>} \\
\hline
\textbf{Description} & Advances the field cursor by one position. Produces no output. If the cursor is already at the last field, this operation is a no-op. \\
\hline
\textbf{Example} & Code and output below. \\
\end{tabularx}}

\vspace{0.3em}
\lstset{language=WBLL}
\begin{lstlisting}
>f
\end{lstlisting}
\textit{Renders:}
\lstset{language=HTML}
\begin{lstlisting}
alice
\end{lstlisting}
\end{tcolorbox}

% \A — attributes for parent <tr> or <li> (non-editing)
\begin{tcolorbox}[tokenbox]
{\small\setstretch{1.05}%
\begin{tabularx}{\linewidth}{@{}p{0.18\linewidth} X@{}}
\textbf{Token} & \wballsyn{\\A('attributes')} \\
\hline
\textbf{Description} & Like \textbf{\textbackslash a}, but applies attributes to the \emph{parent} structural element: the nearest enclosing \texttt{<tr>} (in table contexts) or \texttt{<li>} (in list contexts). Non-editing; produces no output and does \textbf{not} move the field cursor. Has effect only when inside a table row or list item context. \\
\hline
\textbf{Example} & Code and output below. \\
\end{tabularx}}

\vspace{0.3em}
\lstset{language=WBLL}
\begin{lstlisting}
// In a table row context (created by the table subtype)
\A('class="highlight"')e

// In a list item context (created by the list/menu subtype)
\A('data-role="user"')>e(';username')
\end{lstlisting}
\textit{Renders:}
\lstset{language=HTML}
\begin{lstlisting}
<tr class="highlight">... <input type="text" name="price" value="1234.5"> ...</tr>
<li data-role="user">... <input type="text" name="username" value="alice"> ...</li>
\end{lstlisting}
\end{tcolorbox}

% \a — attributes for current element (non-editing)
\begin{tcolorbox}[tokenbox]
{\small\setstretch{1.05}%
\begin{tabularx}{\linewidth}{@{}p{0.18\linewidth} X@{}}
\textbf{Token} & \wballsyn{\\a('attributes')} \\
\hline
\textbf{Description} & Assigns HTML attributes to the most recently emitted element. Use immediately after a token that creates an element (e.g., \texttt{e}, \texttt{w}, \texttt{m}, \texttt{h}). Attributes are appended/merged; later attributes override earlier ones when duplicated. Non-editing; produces no output and does \textbf{not} move the field cursor. \\
\hline
\textbf{Example} & Code and output below. \\
\end{tabularx}}

\vspace{0.3em}
\lstset{language=WBLL}
\begin{lstlisting}
e\a('required style="color:red"')
>e(';username')\a('maxlength="20"')
\end{lstlisting}
\textit{Renders:}
\lstset{language=HTML}
\begin{lstlisting}
<input type="text" name="price" value="1234.5" required style="color:red"><input type="text" name="username" value="alice" maxlength="20">
\end{lstlisting}
\end{tcolorbox}

% \\r — line break (non-editing)
\begin{tcolorbox}[tokenbox]
{\small\setstretch{1.05}%
\begin{tabularx}{\linewidth}{@{}p{0.18\linewidth} X@{}}
\textbf{Token} & \wballsyn{\\r} \\
\hline
\textbf{Description} & Inserts a line break: \texttt{<br>}. Non-editing; produces no value output and does \textbf{not} advance the field cursor. \\
\hline
\textbf{Example} & Code and output below. \\
\end{tabularx}}

\vspace{0.3em}
\lstset{language=WBLL}
\begin{lstlisting}
f\rf
\end{lstlisting}
\textit{Renders:}
\lstset{language=HTML}
\begin{lstlisting}
1234.5<br>alice
\end{lstlisting}
\end{tcolorbox}

% e — text input
\begin{tcolorbox}[tokenbox]
{\small\setstretch{1.05}%
\begin{tabularx}{\linewidth}{@{}p{0.18\linewidth} X@{}}
\textbf{Token} & \wballsyn{e('format[;name][;value]')} \\
\hline
\textbf{Description} & Renders a text input element: \texttt{<input type="text" name="<name>" value="<formatted value>">}. Arguments are \emph{positional but optional}. Defaults: \textit{format} = identity, \textit{name} = active field name, \textit{value} = active field value. \textbf{Cursor rule (editing):} the cursor advances when \textit{name} is not specified (e.g., just \textbf{e}); if \textit{name} is provided, the cursor does not advance (even if \textit{value} is omitted). \textbf{Formatting:} \textit{format} applies only to numeric values. \\
\hline
\textbf{Example} & Code and output below. \\
\end{tabularx}}

\vspace{0.3em}
\lstset{language=WBLL}
\begin{lstlisting}
e('€ #,##0.00;price;1234.5') // All args explicit: no cursor advance; numeric format applied
e // Use active field name and value: advances
>e(';username') // Move to username, explicit name: no cursor advance
\end{lstlisting}
\textit{Renders:}
\lstset{language=HTML}
\begin{lstlisting}
<input type="text" name="price" value="€ 1,234.50"><input type="text" name="price" value="1234.5"><input type="text" name="city" value="Milan">
\end{lstlisting}
\end{tcolorbox}

% f — formatted field value
\begin{tcolorbox}[tokenbox]
{\small\setstretch{1.05}%
\begin{tabularx}{\linewidth}{@{}p{0.18\linewidth} X@{}}
\textbf{Token} & \wballsyn{f('format')} \\
\hline
\textbf{Description} & Renders the active field value at the cursor, optionally applying a numeric \textit{format}. The \textit{format} argument is positional but optional (default: identity). \textbf{Cursor rule (non-editing):} advances because it consumes the active field value. \textbf{Formatting:} applies only to numeric values. \\
\hline
\textbf{Example} & Code and output below. \\
\end{tabularx}}

\vspace{0.3em}
\lstset{language=WBLL}
\begin{lstlisting}
f('€ #,##0.00')\r // Currency formatting of active numeric field
f // No format (identity)
\end{lstlisting}
\textit{Renders:}
\lstset{language=HTML}
\begin{lstlisting}
€ 1,234.50<br>1234.5
\end{lstlisting}
\end{tcolorbox}

% h — hidden input
\begin{tcolorbox}[tokenbox]
{\small\setstretch{1.05}%
\begin{tabularx}{\linewidth}{@{}p{0.18\linewidth} X@{}}
\textbf{Token} & \wballsyn{h('name[;value]')} \\
\hline
\textbf{Description} & Inserts an HTML input element of type hidden. Arguments are \emph{positional but optional}. Defaults when omitted: \textit{name} = active field name, \textit{value} = active field value. \textbf{Cursor rule (editing):} the cursor advances when \textit{name} is not specified (e.g., just \textbf{h}); if \textit{name} is provided, the cursor does not advance. \\
\hline
\textbf{Example} & Code and output below. \\
\end{tabularx}}

\vspace{0.3em}
\lstset{language=WBLL}
\begin{lstlisting}
h
h('kind;area')
h('id')
\end{lstlisting}
\textit{Renders:}
\lstset{language=HTML}
\begin{lstlisting}
<input type="hidden" name="price" value="1234.5"><input type="hidden" name="kind" value="area"><input type="hidden" name="id" value="alice">
\end{lstlisting}
\end{tcolorbox}

% l — active field name (no cursor advance)
\begin{tcolorbox}[tokenbox]
{\small\setstretch{1.05}%
\begin{tabularx}{\linewidth}{@{}p{0.18\linewidth} X@{}}
\textbf{Token} & \wballsyn{l('label')} \\
\hline
\textbf{Description} & Renders a label for the active field's \emph{name}. If a string argument is provided as \wballsyn{'label'}, it overrides the current field name. Produces no field value output and \textbf{does not} advance the cursor. \\
\hline
\textbf{Example} & Code and output below. \\
\end{tabularx}}

\vspace{0.3em}
\lstset{language=WBLL}
\begin{lstlisting}
lf
\rl('User Name')f
\end{lstlisting}
\textit{Renders:}
\lstset{language=HTML}
\begin{lstlisting}
<label>price</label>1234.5<br><label>User Name</label>alice
\end{lstlisting}
\end{tcolorbox}

% m — multiline text area
\begin{tcolorbox}[tokenbox]
{\small\setstretch{1.05}%
\begin{tabularx}{\linewidth}{@{}p{0.18\linewidth} X@{}}
\textbf{Token} & \wballsyn{m('name[;value]')} \\
\hline
\textbf{Description} & Renders a multiline text area: \texttt{<textarea name="<name>"><value></textarea>}. Arguments are \emph{positional but optional}. Defaults: \textit{name} = active field name, \textit{value} = active field value. \textbf{Cursor rule (editing):} advances when \textit{name} is not specified; if \textit{name} is provided, does not advance. \\
\hline
\textbf{Example} & Code and output below. \\
\end{tabularx}}

\vspace{0.3em}
\lstset{language=WBLL}
\begin{lstlisting}
m('comments;Hello, world!')m>m(';New York')
\end{lstlisting}
\textit{Renders:}
\lstset{language=HTML}
\begin{lstlisting}
<textarea name="comments">Hello, world!</textarea><textarea name="price">1234.5</textarea><textarea name="city">New York</textarea>
\end{lstlisting}
\end{tcolorbox}

% \\s — set content attributes (non-editing)
\begin{tcolorbox}[tokenbox]
{\small\setstretch{1.05}%
\begin{tabularx}{\linewidth}{@{}p{0.18\linewidth} X@{}}
\textbf{Token} & \wballsyn{\\s('attr="value" [attr="value"] ...')} \\
\hline
\textbf{Description} & Sets attributes on the current \emph{content} object (for example, a content with subtype \textit{table}, \textit{list}, or \textit{form}). Recognized keys: \texttt{caption}, \texttt{header}, \texttt{footer}, \texttt{key}, \texttt{visible}, \texttt{enabled}, \texttt{disabled}, \texttt{nodata}. Attributes accumulate; later calls override earlier values for the same key. Non-editing; produces no direct output and does \textbf{not} move the field cursor. The visual effect depends on the active content subtype. \\
\hline
\textbf{Example} & Code and output below. \\
\end{tabularx}}

% Keep verbatim listings outside tabularx to prevent runaway arguments
\vspace{0.3em}
\lstset{language=WBLL}
\begin{lstlisting}
// In a table content (subtype: table)
\s('caption="Users" header="Found: @@rows" key="fId"')
lflflf
\end{lstlisting}
\textit{Renders:}
\lstset{language=HTML}
\begin{lstlisting}
<div class="content-header">Found: 3</div>
<table data-key="fId">
  <caption>Users</caption>
  <tr>
    <th>price</th><td>1234.5</td><th>username</th><td>alice</td><th>city</th><td>Milan</td>
  </tr>
  <!-- other rows... -->
</table>
\end{lstlisting}
\end{tcolorbox}

% t — text
\begin{tcolorbox}[tokenbox]
{\small\setstretch{1.05}%
\begin{tabularx}{\linewidth}{@{}p{0.18\linewidth} X@{}}
\textbf{Token} & \wballsyn{t('text')} \\
\hline
\textbf{Description} & Inserts the given text string into the response flow at the current position. \\
\hline
\textbf{Example} & Code and output below. \\
\end{tabularx}}

% Keep verbatim listings outside tabularx to prevent runaway arguments
\vspace{0.3em}
\lstset{language=WBLL}
\begin{lstlisting}
t('Hello World!')
\end{lstlisting}
\textit{Renders:}
\lstset{language=HTML}
\begin{lstlisting}
Hello World!
\end{lstlisting}
\end{tcolorbox}

% w — password input
\begin{tcolorbox}[tokenbox]
{\small\setstretch{1.05}%
\begin{tabularx}{\linewidth}{@{}p{0.18\linewidth} X@{}}
\textbf{Token} & \wballsyn{w('name[;value]')} \\
\hline
\textbf{Description} & Renders a password input element: \texttt{<input type="password" name="<name>" value="<value>">}. Arguments are \emph{positional but optional}. Defaults: \textit{name} = active field name, \textit{value} = active field value. \textbf{Cursor rule (editing):} advances when \textit{name} is not specified; if \textit{name} is provided, does not advance. \textit{Note:} some browsers ignore preset password values. \\
\hline
\textbf{Example} & Code and output below. \\
\end{tabularx}}

\vspace{0.3em}
\lstset{language=WBLL}
\begin{lstlisting}
w('pwd;secret')\rw\rw(';password')
\end{lstlisting}
\textit{Renders:}
\lstset{language=HTML}
\begin{lstlisting}
<input type="password" name="pwd" value="secret"><br><input type="password" name="price" value="1234.5"><input type="password" name="name" value="password">
\end{lstlisting}
\end{tcolorbox}

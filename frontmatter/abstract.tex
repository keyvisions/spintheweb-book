% Abstract for the Spin the Web Book

\chapter*{Abstract}
\addcontentsline{toc}{chapter}{Abstract}

\textbf{Spin the Web} is a framework for building enterprise web portals (``portals'') that virtualize the company brand—\textit{eBranding}. It addresses the persistent challenge of unifying heterogeneous enterprise systems (ERP, CRM, BPMS, and MRP systems) behind a single, role-aware digital channel, providing consistent abstractions over disparate backends.

The framework comprises three core components:
\begin{enumerate}
\item \textbf{Webbase Description Language (WBDL)}: A declarative language for modeling portal structure, content, and behavior.
\item \textbf{Web Spinner}: An engine that interprets webbases (modular portal definitions written in WBDL) to dynamically generate personalized user experiences with real-time content delivery and role-based authorization.
\item \textbf{Spin the Web Studio}: An authoring webbaselet for in-place editing of webbases; it also serves as a laboratory for exercising the Web Spinner.
\end{enumerate}

The project also introduces:
\begin{itemize}
\item \textbf{Webbaselets}: Modular, self-contained WBDL fragments that encapsulate integrations with enterprise systems.
\item \textbf{Webbase Placeholders Language (WBPL)}: A security-conscious templating mechanism for parameterized queries.
\item \textbf{Webbase Layout Language (WBLL)}: A token-based templating language for presentation and rendering.
\item \textbf{Virtualized Company paradigm}: A unified portal interface for customers, employees, suppliers, partners, and governance stakeholders, with the primary objective of digitally harboring the brand (\textit{eBranding}).
\end{itemize}

This book blends foundations with practical guidance for developers, integrators, and technology leaders modernizing digital channels.

\textbf{Keywords:} enterprise portals; WBDL; webbaselets; WBPL; WBLL; eBranding; STW; Foundation.



\clearpage

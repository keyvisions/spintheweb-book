% Define glossary entries
\newglossaryentry{portal}{
    name={portal},
    plural={portals},
    description={An \textbf{Enterprise Web Portal}: a single access point that unifies the capabilities of a website (public information and brand presence) and a web application (aggregation, personalization, discovery, interactive and transactional workflows). In this book, unless otherwise specified, the term \emph{portal} refers to the \textbf{Enterprise Web Portal} and is capable of handling all virtualizable activities associated with an enterprise across roles and channels}
}

\newglossaryentry{rest}{
    name={REST},
    description={Representational State Transfer. A software architectural style that defines a set of constraints to be used for creating Web services}
}

\newglossaryentry{api}{
    name={API},
    description={Application Programming Interface. A set of definitions and protocols for building and integrating application software}
}

\newglossaryentry{uri}{
    name={URI},
    description={Uniform Resource Identifier. A unique sequence of characters that identifies a logical or physical resource}
}

% Project-specific terms
\newglossaryentry{webbase}{
    name={webbase},
    plural={webbases},
    description={A complete WBDL-described portal: the full, declarative structure of a site including areas, pages, contents, datasources, and configuration}
}

\newglossaryentry{webbaselet}{
    name={webbaselet},
    plural={webbaselets},
    description={A modular fragment of a webbase. A WBDL document rooted at an STWArea that can be imported or integrated into a full webbase}
}

\newglossaryentry{ebranding}{
    name={eBranding},
    description={The practice of virtualizing all virtualizable aspects of an entity (organization, enterprise, trade, or group) into a single, coherent digital channel. eBranding unifies documentation and action: it presents public and private information, interactive workflows, and system integrations through a role-aware portal that acts as the brand's primary digital harbor. It emphasizes consistency, accessibility, and governance across audiences (public, customers, suppliers, partners, regulators, employees), and favors integration over replacement by orchestrating APIs and existing systems behind a unified experience}
}

% Chapter: Structuring a Web Portal: A Practical Guide
\chapter{Structuring a Web Portal: A Practical Guide}
\label{chap:portal-structure}

\begin{quote}
\textit{"Structure is not just a means to a solution. It is the solution."} \\
— Alexandra V. Agranovsky
\end{quote}

\section{From Business to Portal Structure}
\label{sec:business-to-portal}

A company's partial or total structure can be projected in a web portal: the lobby, human resources, sales, marketing, production, logistics, administration, service, support… you name it. Each of these areas serves functionalities that address specific needs. Some functionalities satisfy cross-company needs, some are public while others are private, and some expose everything while others only a part based on authorization.

The preceding chapters have detailed the languages and mechanics of the Spin the Web framework. This chapter provides a practical guide on how to translate an organization's business structure and user needs into a coherent and effective web portal using the core \wbdl{} elements: \texttt{STWArea}, \texttt{STWPage}, and \texttt{STWContent}.

The key to a successful portal is a structure that feels intuitive to its users. This is achieved by mirroring the familiar functions of the business itself.

\section{Documentation Through Structure}
\label{sec:documentation-structure}

Before diving into the structural methodology, it's crucial to understand that every \wbdl{} element—from the root \texttt{STWSite} down to individual \texttt{STWContent} blocks—includes localized \texttt{keywords} and \texttt{description} elements. These serve far more than basic metadata; they provide the infrastructure for embedding organizational knowledge directly within the portal structure.

\subsection{Quality Management Integration}
\label{sec:quality-management}

This hierarchical documentation system transforms the portal into a comprehensive organizational knowledge base where quality management principles, procedures, and manuals are seamlessly integrated with the functional structure. Each level serves specific documentation purposes:

\begin{description}
\item[\textbf{Site Level}]: Overall company quality policy, mission statement, and enterprise-wide standards
\item[\textbf{Area Level}]: Department-specific procedures, compliance requirements, and operational standards
\item[\textbf{Page Level}]: Specific process documentation, workflows, and user guidelines
\item[\textbf{Content Level}]: Individual field instructions, data validation rules, and contextual help
\end{description}

\subsection{Living Documentation Benefits}
\label{sec:living-documentation}

This approach provides several critical advantages for enterprise implementation:

\begin{itemize}
\item \textbf{Living Documentation}: Procedures stay current with actual portal functionality, eliminating outdated manuals
\item \textbf{Contextual Access}: Users access relevant quality information exactly where they need it in their workflow
\item \textbf{Compliance Tracking}: Keywords enable systematic auditing and compliance reporting across the organization
\item \textbf{Multi-Language Support}: Quality documentation can be localized for global operations
\item \textbf{Search Integration}: The \studio{} search capabilities can locate quality procedures across the entire portal structure
\item \textbf{Audit Trails}: Every element becomes a potential compliance checkpoint with embedded documentation
\end{itemize}

Consider this example of an Area with embedded quality management documentation:

\begin{lstlisting}[language=XML,caption={Area with Quality Management Documentation},label={lst:quality-area}]
<area _id="qa_control" type="Area">
    <name>
        <text lang="en"><![CDATA[Quality Control]]></text>
    </name>
    <description>
        <text lang="en"><![CDATA[
        This area implements ISO 9001:2015 Section 8.5 - Production and Service Provision.
        All processes documented here follow our Quality Manual QM-2024-Rev3.
        Key Performance Indicators: Defect Rate < 0.1%, Customer Satisfaction > 95%.
        
        Emergency Procedures: In case of quality issues, immediately escalate to 
        Quality Manager (ext. 4001) and follow NC-PROC-001.
        
        Last Updated: 2024-03-15 | Next Review: 2024-09-15
        ]]></text>
    </description>
    <keywords>
        <text lang="en"><![CDATA[ISO9001, quality control, defect tracking, KPI, emergency procedures, compliance, audit]]></text>
    </keywords>
</area>
\end{lstlisting}

This integration ensures that the portal becomes not just a functional interface, but the authoritative source for organizational procedures and quality standards.

\section{Step 1: Define Top-Level Areas}
\label{sec:define-areas}

The first step is to identify the primary functional divisions of the organization. These will become the top-level \textbf{Areas (\texttt{STWArea})} of your portal. Think of these as the main departments or directorates of the company.

A typical manufacturing company might have the following top-level Areas:
\begin{itemize}
    \item \textbf{Sales}: For customers, sales representatives, and channel partners.
    \item \textbf{Administration}: For finance, HR, and internal services.
    \item \textbf{Backoffice}: For logistics, procurement, and supplier management.
    \item \textbf{Technical Office}: For engineering, R\&D, and product support.
    \item \textbf{Products \& Services}: A publicly accessible area showcasing the company's offerings.
\end{itemize}

Each of these would be defined as an \texttt{STWArea} element in your \wbdl{} file, forming the main navigation structure of your portal.

\subsection{Documentation at the Area Level}
\label{sec:area-documentation}

Each Area definition should include comprehensive documentation that serves both operational and compliance purposes. For example, the Sales Area might include:

\begin{lstlisting}[language=XML,caption={Sales Area with Documentation},label={lst:sales-area-docs}]
<area _id="sales" type="Area">
    <name>
        <text lang="en"><![CDATA[Sales]]></text>
    </name>
    <description>
        <text lang="en"><![CDATA[
        Sales Department Operations Center
        
        This area supports the complete sales lifecycle from lead generation to order fulfillment.
        All activities comply with our Customer Relationship Management Policy CRM-POL-001.
        
        Key Processes:
        - Lead qualification (PROC-SALES-001)
        - Quote generation (PROC-SALES-002)
        - Order processing (PROC-SALES-003)
        - Customer support escalation (PROC-SALES-004)
        
        Performance Targets: 
        - Response time to quotes: <24 hours
        - Customer satisfaction: >90%
        - Lead conversion rate: >15%
        
        Department Manager: Jane Smith (ext. 2001)
        Quality Representative: Mark Johnson (ext. 2005)
        ]]></text>
    </description>
    <keywords>
        <text lang="en"><![CDATA[sales, CRM, leads, quotes, orders, customer service, KPI, performance targets]]></text>
    </keywords>
</area>
\end{lstlisting}

This approach ensures that organizational knowledge, procedures, and quality standards are embedded directly within the portal structure, creating a living documentation system that evolves with the business.

\section{Step 2: Design Pages for User Journeys}
\label{sec:design-pages}

With the main Areas defined, the next step is to create the \textbf{Pages (\texttt{STWPage})} within each Area. The design of these pages should be driven by the specific tasks and journeys of your user personas.

Consider the \textbf{Sales Area}. What do different users need to accomplish here?
\begin{itemize}
    \item \textbf{A Customer} might need to:
        \begin{itemize}
            \item View their order history (\textit{Page: "My Orders"}).
            \item Track a shipment (\textit{Page: "Order Tracking"}).
            \item Contact support (\textit{Page: "Support Center"}).
        \end{itemize}
    \item \textbf{A Sales Representative} might need to:
        \begin{itemize}
            \item View their sales dashboard (\textit{Page: "Sales Dashboard"}).
            \item Access customer information (\textit{Page: "Customer Directory"}).
            \item Generate a new quote (\textit{Page: "Quote Generator"}).
        \end{itemize}
\end{itemize}

Each of these user needs translates directly into a \texttt{STWPage} within the "Sales" \texttt{STWArea}. The role-based visibility rules of \wbdl{} ensure that users only see the pages relevant to them.

\subsection{Documentation at the Page Level}
\label{sec:page-documentation}

Pages should document specific workflows and processes. For example, the "Quote Generator" page might include detailed process documentation:

\begin{lstlisting}[language=XML,caption={Page with Process Documentation},label={lst:page-docs}]
<page _id="quote_generator" type="Page">
    <name>
        <text lang="en"><![CDATA[Quote Generator]]></text>
    </name>
    <description>
        <text lang="en"><![CDATA[
        Quote Generation Process - PROC-SALES-002
        
        This page implements the standardized quote generation workflow as defined 
        in our Sales Operations Manual section 4.2.
        
        Required Information:
        1. Customer contact details (validated against CRM)
        2. Product specifications and quantities
        3. Delivery requirements and timeline
        4. Special terms and conditions
        
        Approval Matrix:
        - Quotes up to $10,000: Sales Rep approval
        - Quotes $10,001-$50,000: Sales Manager approval
        - Quotes above $50,000: Sales Director approval
        
        System Integration: Automatically syncs with SAP ERP Quote Module
        Response Time SLA: Quote delivered within 24 hours of request
        ]]></text>
    </description>
    <keywords>
        <text lang="en"><![CDATA[quotes, pricing, approval matrix, SLA, SAP integration, sales process]]></text>
    </keywords>
</page>
\end{lstlisting}

\section{Step 3: Build Pages with Content Blocks}
\label{sec:build-with-content}

The final step is to populate the pages with \textbf{Content (\texttt{STWContent})} blocks. These are the building blocks that display information and provide functionality. A single page will typically be composed of multiple content blocks.

Let's design the public-facing \textbf{"Products"} page from the "Products \& Services" Area. This page needs to be informative and engaging for potential customers. It could be built with the following \texttt{STWContent} blocks:

\begin{enumerate}
    \item \textbf{A Hero Banner}: An eye-catching banner image with a marketing headline. This is a static content block.
    \item \textbf{Product Categories Menu}: A navigation menu, perhaps implemented as tabs or an accordion, that allows users to filter products by category. This content block would use \wbpl{} to dynamically query the list of product categories from a database.
    \item \textbf{Product Listing Grid}: A grid that displays the products. Each item in the grid would show a product image, name, and a short description. This is a highly dynamic content block, driven by a \wbpl{} query that fetches the product data based on the selected category.
    \item \textbf{Featured Products Carousel}: A rotating carousel showcasing new or featured products. This could be driven by a separate, specialized query.
    \item \textbf{Call to Action}: A block encouraging users to "Request a Quote" or "Contact Sales," linking to another page in the portal.
\end{enumerate}

By combining these static and dynamic content blocks, you can create a rich, interactive, and data-driven page that effectively serves the needs of its audience.

\subsection{Documentation at the Content Level}
\label{sec:content-documentation}

Individual content blocks can contain specific field-level instructions and validation rules. For example, a customer information form might include:

\begin{lstlisting}[language=XML,caption={Content with Field-Level Documentation},label={lst:content-docs}]
<content _id="customer_info_form" type="Content">
    <name>
        <text lang="en"><![CDATA[Customer Information]]></text>
    </name>
    <description>
        <text lang="en"><![CDATA[
        Customer Data Collection Form - FORM-CRM-001
        
        This form collects essential customer information required for quote generation.
        All fields marked with (*) are mandatory as per our Customer Data Policy CDP-001.
        
        Data Validation Rules:
        - Company Name: Minimum 2 characters, maximum 100 characters
        - Email: Must be valid business email format
        - Phone: International format preferred (+country code)
        - Industry: Select from predefined NAICS categories
        
        Data Protection Notice: 
        Customer data is processed according to GDPR Article 6(1)(b) for contract performance.
        Data retention period: 7 years from last business interaction.
        
        Quality Check: Sales Rep must verify company information against D&B database.
        ]]></text>
    </description>
    <keywords>
        <text lang="en"><![CDATA[customer data, GDPR, validation rules, data quality, mandatory fields, retention policy]]></text>
    </keywords>
</content>
\end{lstlisting}

This multi-level documentation approach ensures that quality management principles, compliance requirements, and operational procedures are seamlessly integrated throughout the portal structure, creating a comprehensive knowledge management system that supports both daily operations and audit requirements.

\section{Looking Forward}
\label{sec:portal-structure-forward}

This chapter has provided a practical methodology for structuring a web portal. By mapping business functions to Areas, user journeys to Pages, and informational needs to Content blocks, you can design a portal that is both logical in its architecture and intuitive in its use.

In the next chapter, we will examine the specific technology stack used in the reference implementation of the \webspinner{} engine, providing insight into how these declarative structures are translated into a running application.

% Chapter: Structuring a Web Portal: A Practical Guide
\chapter{Structuring a Web Portal: A Practical Guide}
\label{chap:portal-structure}

\begin{quote}
\textit{"Structure is not just a means to a solution. It is the solution."} \\
— Alexandra V. Agranovsky
\end{quote}

\section{From Business to Portal Structure}
\label{sec:business-to-portal}

A company's partial or total structure can be projected in a web portal: the lobby, human resources, sales, marketing, production, logistics, administration, service, support… you name it. Each of these areas serves functionalities that address specific needs. Some functionalities satisfy cross-company needs, some are public while others are private, and some expose everything while others only a part based on authorization.

The preceding chapters have detailed the languages and mechanics of the Spin the Web framework. This chapter provides a practical guide on how to translate an organization's business structure and user needs into a coherent and effective web portal using the core \wbdl{} elements: \texttt{STWArea}, \texttt{STWPage}, and \texttt{STWContent}.

The key to a successful portal is a structure that feels intuitive to its users. This is achieved by mirroring the familiar functions of the business itself.

\section{Step 1: Define Top-Level Areas}
\label{sec:define-areas}

The first step is to identify the primary functional divisions of the organization. These will become the top-level \textbf{Areas (\texttt{STWArea})} of your portal. Think of these as the main departments or directorates of the company.

A typical manufacturing company might have the following top-level Areas:
\begin{itemize}
    \item \textbf{Sales}: For customers, sales representatives, and channel partners.
    \item \textbf{Administration}: For finance, HR, and internal services.
    \item \textbf{Backoffice}: For logistics, procurement, and supplier management.
    \item \textbf{Technical Office}: For engineering, R\&D, and product support.
    \item \textbf{Products \& Services}: A publicly accessible area showcasing the company's offerings.
\end{itemize}

Each of these would be defined as an \texttt{STWArea} element in your \wbdl{} file, forming the main navigation structure of your portal.

\section{Step 2: Design Pages for User Journeys}
\label{sec:design-pages}

With the main Areas defined, the next step is to create the \textbf{Pages (\texttt{STWPage})} within each Area. The design of these pages should be driven by the specific tasks and journeys of your user personas.

Consider the \textbf{Sales Area}. What do different users need to accomplish here?
\begin{itemize}
    \item \textbf{A Customer} might need to:
        \begin{itemize}
            \item View their order history (\textit{Page: "My Orders"}).
            \item Track a shipment (\textit{Page: "Order Tracking"}).
            \item Contact support (\textit{Page: "Support Center"}).
        \end{itemize}
    \item \textbf{A Sales Representative} might need to:
        \begin{itemize}
            \item View their sales dashboard (\textit{Page: "Sales Dashboard"}).
            \item Access customer information (\textit{Page: "Customer Directory"}).
            \item Generate a new quote (\textit{Page: "Quote Generator"}).
        \end{itemize}
\end{itemize}

Each of these user needs translates directly into a \texttt{STWPage} within the "Sales" \texttt{STWArea}. The role-based visibility rules of \wbdl{} ensure that users only see the pages relevant to them.

\section{Step 3: Build Pages with Content Blocks}
\label{sec:build-with-content}

The final step is to populate the pages with \textbf{Content (\texttt{STWContent})} blocks. These are the building blocks that display information and provide functionality. A single page will typically be composed of multiple content blocks.

Let's design the public-facing \textbf{"Products"} page from the "Products \& Services" Area. This page needs to be informative and engaging for potential customers. It could be built with the following \texttt{STWContent} blocks:

\begin{enumerate}
    \item \textbf{A Hero Banner}: An eye-catching banner image with a marketing headline. This is a static content block.
    \item \textbf{Product Categories Menu}: A navigation menu, perhaps implemented as tabs or an accordion, that allows users to filter products by category. This content block would use \wbpl{} to dynamically query the list of product categories from a database.
    \item \textbf{Product Listing Grid}: A grid that displays the products. Each item in the grid would show a product image, name, and a short description. This is a highly dynamic content block, driven by a \wbpl{} query that fetches the product data based on the selected category.
    \item \textbf{Featured Products Carousel}: A rotating carousel showcasing new or featured products. This could be driven by a separate, specialized query.
    \item \textbf{Call to Action}: A block encouraging users to "Request a Quote" or "Contact Sales," linking to another page in the portal.
\end{enumerate}

By combining these static and dynamic content blocks, you can create a rich, interactive, and data-driven page that effectively serves the needs of its audience.

\section{Looking Forward}
\label{sec:portal-structure-forward}

This chapter has provided a practical methodology for structuring a web portal. By mapping business functions to Areas, user journeys to Pages, and informational needs to Content blocks, you can design a portal that is both logical in its architecture and intuitive in its use.

In the next chapter, we will examine the specific technology stack used in the reference implementation of the \webspinner{} engine, providing insight into how these declarative structures are translated into a running application.

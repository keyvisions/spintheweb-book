% Part III: The Web Portal

\chapter*{Introduction to Part III: The Web Portal}
\addcontentsline{toc}{chapter}{Introduction to Part III: The Web Portal}
\label{part:implementation}

\begin{quote}
\textit{"Structure is not just a means to a solution. It is the solution."} \\
— Alexandra V. Agranovsky
\end{quote}

With the Spin the Web framework fully specified in Part II, this part transitions from engineering blueprints to practical application. It serves as the developer's guide to the models and methodologies for *using* the framework to design, build, and structure a real-world enterprise portal.

This part now divides implementation into two complementary dimensions:
\begin{itemize}
	\item \textbf{Portal Contents (Structure, Semantics, Information)} — Modeling Areas, Pages, and Content blocks; embedding documentation; designing navigation, search, and task-oriented information architecture (\cref{chap:portal-contents}).
	\item \textbf{Portal Visuals (Presentation, Layout, Interaction)} — Translating structure into accessible, brand-aligned, performant user interfaces while preserving semantic integrity (\cref{chap:portal-visuals}).
\end{itemize}

Throughout, we reference the public website \href{https://spintheweb.org}{spintheweb.org} and the companion \textit{Spin the Web Studio}—both built with the framework—as running examples of the patterns described here. For an overview of the community catalog used to discover and rate webbaselets, see the Ecosystem webbaselet in \cref{sec:app-ecosystem}.

By the end of this part, you will have a clear methodology for designing and building sophisticated web portals using the Spin the Web framework—treating structure and presentation as orthogonal layers that evolve independently yet harmonize in the delivered experience.

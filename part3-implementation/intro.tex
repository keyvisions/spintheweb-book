% Part III: Implementation and Practice

\chapter*{Introduction to Part III}
\addcontentsline{toc}{chapter}{Introduction to Part III}
\label{part:implementation}

This part focuses on practical implementation details, guiding you from conceptual models to running systems. It bridges the specification layer to concrete engineering choices across language, runtime, tooling, and deployment.

We cover the selected technology stack, module boundaries, build strategy, and the mechanics of transforming WBDL/WBPL inputs into production-grade web portals. Particular attention is given to developer experience, testability, observability, and security.

\begin{description}
\item[\textbf{Chapter 11: Structuring a Web Portal}] (\cref{chap:portal-structure}) -- A practical guide to designing a portal's structure based on business needs and user journeys.

\item[\textbf{Chapter 12: Practical Learning Path}] (\cref{chap:learning}) -- A pragmatic roadmap for practitioners to move from beginner to proficient contributor.

\item[\textbf{Chapter 13: Technology Stack}] (\cref{chap:implementation-tech}) -- The end-to-end implementation view: Deno/TypeScript, module layout, parsing and execution, session/state handling, integration points, and delivery.
\end{description}

By the end of this part, you should be able to implement, run, and iterate on a Spin the Web system with confidence.

% Part III: Implementation
\chapter{Technology Stack and Implementation Mechanics}
\label{chap:implementation-tech}

This chapter documents the concrete technologies used to implement the web spinner mechanics and how the theoretical constructs from Part II are realized in a working system.

\section{Runtime and Languages}
The reference implementation is written in TypeScript and runs on the Deno runtime:\footnote{See the project repository: https://github.com/keyvisions/spintheweb}
\begin{itemize}
	\item Deno 2.x runtime and TypeScript for server-side logic
	\item Standard Web APIs (URL, URLSearchParams, Fetch, Crypto) leveraged directly in server code
	\item No Node.js dependency; tasks and scripts are run via Deno (e.g., CSS/JS minification)
	\item Containerization via a multi-stage Dockerfile based on denoland/deno:alpine
\end{itemize}

\section{High-Level Architecture}
The spinner is a server that understands WBML. On each request it:\footnote{See also the ``Paradigm'' section in the implementation README}
\begin{enumerate}
	\item Establishes or resumes a session (user, roles, locale, placeholders)
	\item Ensures the requested WBML/Webbase is loaded into an in-memory tree
	\item Decides whether to respond with a resource directly or with a list of REST calls the client should make (async via WebSockets)
	\item Renders contents on demand using WBLL-driven layouts and returns HTML fragments/resources
\end{enumerate}

\section{Core Elements and Site Tree}
The in-memory model mirrors the WBML structure:
\begin{itemize}
	\item \textbf{STWSite} (singleton root), \textbf{STWArea}, \textbf{STWPage}, \textbf{STWContent}
	\item All derive from an abstract \textbf{STWElement} providing identity, naming, localization, hierarchy, and export to WBML
	\item A content type (e.g., Text, Table, Menus, Breadcrumbs, Calendar, Code Editor, ...), implemented under \texttt{stwContents/}, encapsulates data access and rendering concerns
\end{itemize}

\section{Rendering Pipeline: WBLL and WBPL}
Presentation is described with WBLL (Webbase Layout Language), interpreted by a layout engine:
\begin{itemize}
	\item WBLL strings are tokenized and validated; tokens drive generation of a specialized render function
	\item Token handlers cover inputs, lists, links, media, buttons/actions, and structural fragments; they build HTML using placeholders and field values
	\item WBPL expressions provide string interpolation and conditional/functional logic within layouts and settings
	\item Placeholders (e.g., \verb|@@name|) merge session, request, and record values
\end{itemize}

\section{Request Flow in Practice}
For a content render request:
\begin{enumerate}
	\item The session determines visibility (role-based) and language
	\item The content locates its WBLL layout for the current language
	\item Records are fetched (via a datasource or parameters); the first row and fields hydrate placeholders
	\item The compiled WBLL render function executes, producing the HTML body; optional header/footer wrappers apply
\end{enumerate}

\section{Security, Localization, and State}
\begin{itemize}
	\item \textbf{Visibility}: role-based flags inherited along the element tree control exposure of nodes
	\item \textbf{Localization}: localized properties (names, slugs, messages) are resolved through the session
	\item \textbf{State}: per-session placeholders and content-level settings influence rendering and actions
\end{itemize}

\section{Build, Tooling, and Deployment}
\begin{itemize}
	\item Deno tasks: merge and minify static assets (e.g., CSS merger, JS minifier) for the \texttt{public/} client assets
	\item Tests: parsing and evaluation tests for WBPL ensure correctness of expressions and escaping
	\item Docker: multi-stage build caches Deno dependencies and ships a non-root runtime image
\end{itemize}

\section{Where the Mechanics Live (Guide to Source)}
The following folders in the reference implementation contain the mechanics described above:
\begin{itemize}
	\item \texttt{stwElements/}: \textit{STWElement}, \textit{STWSite}, \textit{STWArea}, \textit{STWPage}, \textit{STWContent}
	\item \texttt{stwContents/}: concrete contents and \textit{WBLL} engine (layout parsing/rendering)
	\item \texttt{tests/}: WBPL and layout-related unit tests
	\item \texttt{public/}: client-side scripts, styles, and SPA shell
	\item \texttt{tasks/}: Deno-powered dev/build utilities (e.g., minification, CSS merge)
\end{itemize}

\section{Example: From WBML to HTML}
At a glance:
\begin{enumerate}
	\item WBML defines the site tree; the spinner builds an in-memory model at startup/load
	\item A user navigates to a page; the spinner locates the route and the associated contents
	\item Each content loads data (if needed), prepares placeholders, and renders its WBLL layout
	\item The server responds with an HTML fragment or instructs the client to fetch multiple fragments via REST/WebSockets
\end{enumerate}

\noindent This chapter bridges theory and implementation so future chapters can delve into deployment topologies, performance tuning, and advanced examples.
